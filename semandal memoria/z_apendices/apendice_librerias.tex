\chapter{Librerías y software utilizado}
\label{anexo:libreria}
A continuación se muestran las librerías y software necesarios para hacer funcionar el proyecto Semandal.
\section{Librerías}
\subsection{Python}
\subsubsection{re}
Librería para resolver expresiones regulares.
Esta librería se incluye en la instalación de Python.
\subsubsection{BeautifulSoup}
BeautisulSoup es una de las librerías más populares para trabajar con código HTML en Python.
Se puede descargar desde la web de BeautifulSoup\footnote{\url{http://www.crummy.com/software/BeautifulSoup/}}.
\subsubsection{urllib2}
Librería para acceder a un web a través de una URL y obtener el código HTML de la página que se pide.
Esta librería se incluye en la instalación de Python.
\subsubsection{xlrd}
Librería que permite leer ficheros de Excel (.xls) Se puede descargar de forma gratuita desde la web oficial de Python\footnote{\url{https://pypi.python.org/pypi/xlrd}}.
\subsubsection{MySQLdb}
Librería que sirve de interfaz con una base de datos MySQL. Se puede descargar de forma gratuita desde la web oficial de Python\footnote{\url{https://pypi.python.org/pypi/MySQL-python}}.
\subsubsection{subprocess}
Librería para generar y lanzar nuevos procesos recuperando la salida de estos.
Esta librería se incluye en la instalación de Python.
\subsubsection{datetime}
Librería para trabajar con fechas.
Esta librería se incluye en la instalación de Python.
\subsubsection{sys}
Librería para interactuar con el interprete de Python.
Esta librería se incluye en la instalación de Python.
\subsubsection{os}
Librería para interactuar con el sistema operativo.
Esta librería se incluye en la instalación de Python.

\subsection{Java}
\subsubsection{JESS}
Motor de reglas y framework de razonamiento escrito en Java. Se puede integrar en aplicaciones escritas en Java y Android. Se puede obtener una licencia gratuita con fines educativos en la web de JESS\footnote{\url{http://herzberg.ca.sandia.gov/}}.
\section{Software}
\subsection{Django}
Framework de desarrollo escrito en Python, utilizado para la creación y mantenimiento de la API, se puede descargar de forma gratuita desde la web de Django\footnote{\url{https://www.djangoproject.com/download/}}.
\subsubsection{Django-extensions}
Django-extension es un conjunto de comandos y extras que se añaden a la funcionalidad que ofrece Django, en nuestro caso la hemos utilizado para realizar un diagrama de base de datos. Se pueden descargar desde la web oficial de Python\footnote{\url{https://pypi.python.org/pypi/django-extensions/1.5.0}}.

\subsection{Eclipse}
Entorno de desarrollo para Java, utilizado para la obtención de reglas y la clasificación. Puede descargarse desde la web de Eclipse\footnote{\url{https://eclipse.org/downloads/}}.

\subsection{Eclipse (Android SDK)}
Entorno de desarrollo para Java orientado al desarrollo en Android. Utilizado para el desarrollo de la aplicación en Android. También nos instala los emuladores necesarios para probar nuestro código en Android. Se puede descargar desde la web de Android\footnote{\url{http://developer.android.com/sdk/index.html}}.

\subsection{Concept Explorer}
Aplicación para trabajar con Análisis formal de conceptos. Se puede descargar gratuitamente desde la web de Concept Explorer\footnote{\url{http://conexp.sourceforge.net/}}.
\subsection{WEKA}
Suite de Inteligencia artificial y minería de datos utilizada para trabajar con algoritmos C4.5 y RIPPER. Se puede descargar de forma gratuita desde la web de WEKA\footnote{\url{http://www.cs.waikato.ac.nz/ml/weka/}}.

\subsection{Graphviz}
Aplicación para generar grafos a partir del lenguaje dot, utilizada junto a las django-extensions para generar el diagrama de la base de datos. Se puede descargar desde la web de Graphviz\footnote{\url{http://www.graphviz.org/Download..php}}.